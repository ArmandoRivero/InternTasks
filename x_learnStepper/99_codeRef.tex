%******************************************************************************
\section{Code Reference}
%******************************************************************************
This section describe a reference for the code, it is written in the file "99$\_$codeRef.tex", this can be removed from he "main.tex" once the document content is writen in "1$\_$content.tex" file.

%%******************************************************************************
%\section{Figures and bibliography}
%%******************************************************************************
%------------------------------------------------------------------------------
\subsection{Figures}
%------------------------------------------------------------------------------
A figure is included like this:

\begin{figure}[H]
\begin{center}
 \includegraphics[scale=0.5]{images/latex}
\caption{caption is here}
\label{fig: latex}
\end{center}
\end{figure}


....and referenced like: figure \ref{fig: latex}. Use a folder named "images" in the project's directory for all the images used. This keeps order with the files.\\



%------------------------------------------------------------------------------
\subsection{Reference bibliography}
%------------------------------------------------------------------------------

To reference from the Bibliography use: for a custom citation \cite{adams1995hitchhiker} or another kind of citation \cite{myRefrenceCreated}, for a paper \cite{4690904}. The bibliography reference information is placed in a file named "bibliography.bib" \\


%------------------------------------------------------------------------------
\subsection{Special characters}
%------------------------------------------------------------------------------
Temperature can be written as 35 \degree C, as well as resistance 470 \ohm ....\\



%%******************************************************************************
%\section{Links, tables and items}
%%******************************************************************************
%------------------------------------------------------------------------------
\subsection{Links}
%------------------------------------------------------------------------------
To know more about links visit this link clicking \href{https://en.wikibooks.org/wiki/LaTeX/Hyperlinks}{here}.


%------------------------------------------------------------------------------
\subsection{Tables}
%------------------------------------------------------------------------------

Here is how a table is created and how to reference it as: table \ref{tab: USB characteristics}.

%\begin{landscape}
\begin{table}[h]
  \centering
  \begin{tabular}{|l|l|}
    \hline
    Product ID & 0x6001 \\
    \hline
    Vendor ID & 0x0403 \\
    \hline
  \end{tabular}
  \caption{USB characteristics}
  \label{tab: USB characteristics}
\end{table}
%\end{landscape}




%------------------------------------------------------------------------------
\subsection{Items}
%------------------------------------------------------------------------------
The USB driver must implement the following features: (this is a "itemize" format)

\begin{itemize}
	\item Device may be opened
	\item Baud rate configuration
	\item Parity configuration
	\item Flow control configuration
	\item Transmission and reception of data
\end{itemize}


Another way to represent a list is using "Outlines".

\begin{outline}
	\1 Outline level 1
		\2 outline level 2
			\3 outline level 3
	\1 outline level$_{1}$
		\2 outline level$_{2}$
			\3 outline level$_{3}$
\end{outline}


Using multi-columns to represent a list:
\begin{multicols}{2}
\begin{itemize}
	\item Device may be opened
	\item Baud rate configuration
	\item Parity configuration
	\item Flow control configuration
	\item Transmission and reception of data
\end{itemize}
\end{multicols}



%------------------------------------------------------------------------------
\subsection{Equations}
%------------------------------------------------------------------------------
This is an Equation and it is references as eq. \ref{eq: equation}.

\begin{equation}
\label{eq: equation}
	N_{Total\_total} \ [unit] = \frac{Something_{anIndexWritesHere}}{\Delta Var\_x}
\end{equation}\\



%------------------------------------------------------------------------------
\subsection{Karnaugh maps}
%------------------------------------------------------------------------------


\begin{karnaugh-map}
   \manualterms{1,0,0,1,0,0,0,0,1,1,1,1,0,1,1,1}
   \implicant{13}{11}
   \implicant{15}{10}
   \implicantedge{0}{0}{8}{8}
   \implicantedge{3}{3}{11}{11}
\end{karnaugh-map}







